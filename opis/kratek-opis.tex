\documentclass[a4paper,12pt]{article}
\usepackage[slovene]{babel}
\usepackage[utf8]{inputenc}
\usepackage[T1]{fontenc}
\usepackage{lmodern}
\usepackage{amsmath}
\usepackage{amssymb}
\usepackage[shortlabels]{enumitem}
\usepackage{graphicx}

\newtheorem{definition}{Definicija}

\pagestyle{plain}

\begin{document}
\author{Lovro Verk in Matevž Kusterle}
\date{December 2023}
\title{Weak Edge k-Metric Dimension}
\maketitle

\section{Definicije}

    \begin{definition}
       Naj bo $S \subseteq V(G)$ in $a, b \in V(G) \cup E(G)$. Definiramo $\Delta_S (a,b)$ kot vsoto razlik razdalj od $a$ in $b$ do vsakega vozljišča $S$. Torej je $$\Delta_S (a,b) = \sum_{s \in S } |d(s,a) - d(s,b)|$$.
    \end{definition}

    \begin{definition}
        \textbf{Šibka k-metrična dimenzija na povezavah} grafa $G$ $wedim_k(G)$, je velikost/moč/kardinalno število
        najmanjše podmnožice $S$ grafa $G$, tako da za vsak par povezav $e,f \in E(G)$ velja $\Delta_S (e,f) \geq k$.
    \end{definition}

   
\pagebreak

\section{Problem} 
Za več vrst različnih grafov morava ugotoviti šibko k-metrično dimenzijo na povezavah in pri tem določiti največjo možno vrednost k. Iz dobljenih rezultatov bo potrebno razbrati formule za dimenzije posameznih vrst grafov. Kasneje pa bova poiskala grafe, za katere se šibka k-metrična dimenzija na povezavah razlikuje od navadne šibke k-metrične dimenzije na povezavah.
    
\section{Načrt dela}
Najprej bova implementirala sledeče:
    \begin{itemize}
        \item funkcijo, ki sprejme graf $G$ in vrednost $k$, ter vrne šibko k-metrično dimenzijo na povezavah grafa
        \item funkcijo, ki določi največjo vrednost $k$ grafa $G$
        \item funkcijo, ki sprejme graf $G$ in vrednost $k$, ter vrne šibko k-metrično dimenzijo grafa, da bomo primerjali dimenzije
    \end{itemize}

\end{document}
